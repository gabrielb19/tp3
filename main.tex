\documentclass{article}
\usepackage{xcolor}
\usepackage{graphicx} % Required for inserting images
\usepackage{subcaption}
\usepackage{placeins}
\usepackage{float}
\usepackage{listings} 

\lstset{
    language=Python,
    backgroundcolor=\color{gray!10},
    basicstyle=\ttfamily\small,
    keywordstyle=\color{blue},
    stringstyle=\color{red!60!black},
    commentstyle=\color{green!50!black}\itshape,
    showstringspaces=false,
    frame=single,
    numbers=left,
    numberstyle=\tiny\color{gray},
    breaklines=true,
    captionpos=b
}

\title{tarea_3_pdi}
\author{diegonm0207 }
\date{November 2025}

\usepackage[backend=biber,style=apa]{biblatex}
\addbibresource{bibliography.bib}

\begin{document}


\section{Entrenamiento de ResNet, VGG16 y ViT}



\subsection{Metodología}

\begin{lstlisting}[language=Python]
asd
\end{lstlisting}

\subsection{Resultados}

\begin{figure}[H]
    \centering
    \includegraphics[width=0.5\linewidth]{imgs/1_train_loss_resnet.png}
    \caption{Curvas del training loss y validation loss para el entrenamiento de la red ResNet sobre el conjunto de datos de cataratas.}
    \label{fig:resnet_loss}
\end{figure}

\begin{figure}[H]
    \centering
    \includegraphics[width=0.5\linewidth]{imgs/1_confusion_matrix_resnet.png}
    \caption{Matriz de confusión para el conjunto de validación del modelo ResNet entrenado sobre el conjunto de datos de cataratas.}
    \label{fig:resnet_confusion}
\end{figure}

\begin{figure}[H]
    \centering
    \includegraphics[width=0.5\linewidth]{imgs/1_train_loss_vgg.png}
    \caption{Curvas del training loss y validation loss para el entrenamiento de la red VGG sobre el conjunto de datos de cataratas.}
    \label{fig:vgg_loss}
\end{figure}

\begin{figure}[H]
    \centering
    \includegraphics[width=0.5\linewidth]{imgs/1_confusion_matrix_vgg.png}
    \caption{Matriz de confusión para el conjunto de validación del modelo VGG entrenado sobre el conjunto de datos de cataratas.}
    \label{fig:vgg_confusion}
\end{figure}

\begin{figure}[H]
    \centering
    \includegraphics[width=0.5\linewidth]{imgs/1_train_loss_vit.png}
    \caption{Curvas del training loss y validation loss para el entrenamiento de la red ViT sobre el conjunto de datos de cataratas.}
    \label{fig:vit_loss}
\end{figure}

\begin{figure}[H]
    \centering
    \includegraphics[width=0.5\linewidth]{imgs/1_confusion_matrix_vit.png}
    \caption{Matriz de confusión para el conjunto de validación del modelo ViT entrenado sobre el conjunto de datos de cataratas.}
    \label{fig:vit_confusion}
\end{figure}

\begin{table}[h]
\centering
\begin{tabular}{lcc}
\hline
\textbf{Modelo} & \textbf{F1 Train} & \textbf{F1 Validation} \\
\hline
ResNet & 0.9916 & 0.9774 \\
VGG    & 0.9916 & 1.0000 \\
ViT    & 0.9874 & 1.0000 \\
\hline
\end{tabular}
\caption{Resultados de F1-score para los modelos evaluados sobre el conjunto de datos de cataratas.}
\label{tab:resultados_f1}
\end{table}

\FloatBarrier

\section{2}

\subsection{Resnet}

\begin{table}[h]
\centering
\begin{tabular}{l c}
\hline
\textbf{Split} & \textbf{F1-score} \\
\hline
Split 1  & 0.9324 \\
Split 2  & 0.9199 \\
Split 3  & 0.9081 \\
Split 4  & 0.8668 \\
Split 5  & 0.9205 \\
Split 6  & 0.9149 \\
Split 7  & 0.9163 \\
Split 8  & 0.9424 \\
Split 9  & 0.9577 \\
Split 10 & 0.8892 \\
\hline
\textbf{Promedio} & \textbf{0.9168} \\
\textbf{Desviación estándar} & \textbf{0.0244} \\
\hline
\end{tabular}
\caption{Resultados de F1-score para los 10 splits del modelo resnet.}
\label{tab:f1_resnet_splits}
\end{table}

\begin{figure}[H][h]
    \centering

    \begin{subfigure}{0.48\linewidth}
        \centering
        \includegraphics[width=\linewidth]{imgs/2_train_resnet.png}
        \caption{Curvas de pérdida en el entrenamiento.}
        \label{fig:resnet_train_splits}
    \end{subfigure}
    \hfill
    \begin{subfigure}{0.48\linewidth}
        \centering
        \includegraphics[width=\linewidth]{imgs/2_loss_resnet.png}
        \caption{Curvas de pérdida en el conjunto de validación.}
        \label{fig:resnet_val_splits}
    \end{subfigure}

    \caption{Resultados de entrenamiento y pérdida del modelo ResNet sobre el conjunto de datos de cataratas.}
    \label{fig:resnet_train_val_splits}
\end{figure}

\begin{figure}[H]
    \centering
    \includegraphics[width=\linewidth]{imgs/2_matrix_resnet.png}
    \caption{Matriz de confusión para las particiones de ResNet}
    \label{fig:2_matrix_resnet}
\end{figure}

\FloatBarrier

\subsection{VGG}

\begin{table}[h]
\centering
\begin{tabular}{l c}
\hline
\textbf{Split} & \textbf{F1-score} \\
\hline
Split 1  & 0.8551 \\
Split 2  & 0.8492 \\
Split 3  & 0.9202 \\
Split 4  & 0.9017 \\
Split 5  & 0.9539 \\
Split 6  & 0.8111 \\
Split 7  & 0.8967 \\
Split 8  & 0.8714 \\
Split 9  & 0.9542 \\
Split 10 & 0.8840 \\
\hline
\textbf{Promedio} & \textbf{0.8898} \\
\textbf{Desviación estándar} & \textbf{0.0434} \\
\hline
\end{tabular}
\caption{Resultados de F1-score para los 10 splits del modelo VGG.}
\label{tab:f1_splits_modelo2}
\end{table}

\begin{figure}[H][h]
    \centering

    \begin{subfigure}{0.48\linewidth}
        \centering
        \includegraphics[width=\linewidth]{imgs/2_train_vgg.png}
        \caption{Curvas de pérdida en el entrenamiento.}
        \label{fig:resnet_train_splits}
    \end{subfigure}
    \hfill
    \begin{subfigure}{0.48\linewidth}
        \centering
        \includegraphics[width=\linewidth]{imgs/2_loss_vgg.png}
        \caption{Curvas de pérdida en el conjunto de validación.}
        \label{fig:resnet_val_splits}
    \end{subfigure}

    \caption{Resultados de entrenamiento y pérdida del modelo VGG sobre el conjunto de datos de cataratas.}
    \label{fig:resnet_train_val_splits}
\end{figure}


\begin{figure}[H]
    \centering
    \includegraphics[width=\linewidth]{imgs/2_matrix_vgg.png}
    \caption{Matriz de confusión para las particiones de VGG}
    \label{fig:2_matrix_vgg}
\end{figure}
\FloatBarrier
\subsection{Vit}

\begin{table}[h]
\centering
\begin{tabular}{l c}
\hline
\textbf{Split} & \textbf{F1-score} \\
\hline
Split 1  & 0.8380 \\
Split 2  & 0.8869 \\
Split 3  & 0.7945 \\
Split 4  & 0.7397 \\
Split 5  & 0.8321 \\
Split 6  & 0.8618 \\
Split 7  & 0.8739 \\
Split 8  & 0.6777 \\
Split 9  & 0.7807 \\
Split 10 & 0.6040 \\
\hline
\textbf{Promedio} & \textbf{0.7889} \\
\textbf{Desviación estándar} & \textbf{0.0869} \\
\hline
\end{tabular}
\caption{Resultados de F1-score para los 10 splits del modelo ViT.}
\label{tab:f1_splits_vit}
\end{table}

\begin{figure}[H][h]
    \centering

    \begin{subfigure}{0.48\linewidth}
        \centering
        \includegraphics[width=\linewidth]{imgs/2_train_vit.png}
        \caption{Curvas de pérdida en el entrenamiento.}
        \label{fig:vit_train_splits}
    \end{subfigure}
    \hfill
    \begin{subfigure}{0.48\linewidth}
        \centering
        \includegraphics[width=\linewidth]{imgs/2_loss_vit.png}
        \caption{Curvas de pérdida en el conjunto de validación.}
        \label{fig:vit_val_splits}
    \end{subfigure}

    \caption{Resultados de entrenamiento y pérdida del modelo ViT sobre el conjunto de datos de cataratas.}
    \label{fig:vit_train_val_splits}
\end{figure}

\begin{figure}[H]
    \centering
    \includegraphics[width=\linewidth]{imgs/2_matrix_vit.png}
    \caption{Matriz de confusión para las particiones de ViT}
    \label{fig:2_matrix_vit}
\end{figure}

\section{Estimadores de incertidumbre}
\subsection{Feature Densities}
\subsection{Distancia de Jensen-Shannon}
\subsection{Deep Ensembles}
\subsection{Monte-Carlo dropout}
\subsection{Distancia de Mahalanobis}

\printbibliography
\end{document}
